%%%%%%%%%%%%%%%%%%%%%%%%%%%%%%%%%%%%%%%%%
% "ModernCV" CV and Cover Letter
% LaTeX Template
% Version 1.1 (9/12/12)
%
% This template has been downloaded from:
% http://www.LaTeXTemplates.com
%
% Original author:
% Xavier Danaux (xdanaux@gmail.com)
%
% License:
% CC BY-NC-SA 3.0 (http://creativecommons.org/licenses/by-nc-sa/3.0/)
%
% Important note:
% This template requires the moderncv.cls and .sty files to be in the same 
% directory as this .tex file. These files provide the resume style and themes 
% used for structuring the document.
%
%%%%%%%%%%%%%%%%%%%%%%%%%%%%%%%%%%%%%%%%%

%----------------------------------------------------------------------------------------
%	PACKAGES AND OTHER DOCUMENT CONFIGURATIONS
%----------------------------------------------------------------------------------------

\documentclass[11pt,a4paper,sans]{moderncv} % Font sizes: 10, 11, or 12; paper sizes: a4paper, letterpaper, a5paper, legalpaper, executivepaper or landscape; font families: sans or roman

\moderncvstyle{casual} % CV theme - options include: 'casual' (default), 'classic', 'oldstyle' and 'banking'
\moderncvcolor{blue} % CV color - options include: 'blue' (default), 'orange', 'green', 'red', 'purple', 'grey' and 'black'

\usepackage{lipsum} % Used for inserting dummy 'Lorem ipsum' text into the template
\usepackage[scale=0.75]{geometry} % Reduce document margins
%\setlength{\hintscolumnwidth}{3cm} % Uncomment to change the width of the dates column
%\setlength{\makecvtitlenamewidth}{10cm} % For the 'classic' style, uncomment to adjust the width of the space allocated to your name

%----------------------------------------------------------------------------------------
%	NAME AND CONTACT INFORMATION SECTION
%----------------------------------------------------------------------------------------

\firstname{James} % Your first name
\familyname{Boocock} % Your last name
% All information in this block is optional, comment out any lines you don't need
\title{Curriculum Vitae}
\address{Apartment 3, 161 High Street}{Dunedin, New Zealand}
\mobile{+64221201361}
\email{smilefreak@gmx.com}

%----------------------------------------------------------------------------------------

\begin{document}

\makecvtitle % Print the CV title

%----------------------------------------------------------------------------------------
%	EDUCATION SECTION
%----------------------------------------------------------------------------------------

\section{Education}

\cventry{2012--2013}{Diploma for Graduates}{The University of Otago}{Dunedin}{NZ}{Genetics and Statistics}  % Arguments not required can be left empty
\cventry{2008--2012}{Bachelor of Science}{The University of Otago}{Dunedin}{NZ}{Computer Science}

%----------------------------------------------------------------------------------------
%	WORK EXPERIENCE SECTION
%----------------------------------------------------------------------------------------

\section{Programming Skills}

\cvitem{Basic}{\textsc{SQL},Amazon API Tools}
\cvitem{Intermediate}{\textsc{Java},\textsc{C},\textsc{C++},\textsc{R},Linux System Administration, Parralel Programming,IBM Load Leveler,Statistics,\textsc{Git},Bioinformatics Tools}
\cvitem{Advanced}{Python, Unix Command-line}
\section{Work Experience}

\cventry{2013--Present}{Assosciate Research Fellow}{Biochemistry Department}{University of Otago}{}{Investigating Copy-number variation in plant and human DNA sequence. 
\\
\\
Current Work
\begin{itemize}
\item Developed a custom XML-RPC server using python for processing data from the 1000 Genomes Project on  Amazon EC2 for further processing on local HPC instances. 
\item Used the R programming language to analyse copy number variation in apple DNA sequence data.
\item Processing large genomic data sets using Unix command-line tools.
\end{itemize}}

%------------------------------------------------

\cventry{2012--2013}{Summer Studentship}{Summer Of eResearch}{}{University of Otago and NZ eScience Infrastructure (NESI)}{Worked on an interface between galaxy and Globus using Python. Other work from the project involved developing a bioinformatics pipeline for analyses of selection signatures in genomic data. The pipeline is currently in preparation for publication and can be found at the url \httplink{https://github.com/smilefreak/MerrimanSelectionPipeline}
\\
\\
Outputs
\begin{itemize}
\item Developed a prototype version of a Galaxy Globus interface using grython a jython version of Grisu which is an open source Java framework designed to sit on top of grid middleware. 
\item Parralelisation of genomic selection tools for standard use, and development of scripts to take advantage of the NeSI IBM load leveler cluster.
\item Python program to run genomic selection tools and process the intermediate data files.
\end{itemize}}
\cventry{2012--2013}{Summer Studentship}{Summer Of eResearch}{}{University of Otago and NZ eScience Infrastructure (NeSI)}{
The project focused on maintaining a local galaxy bioinformatics instance. Added a large amount of tools to the local galaxy instance (galaxy is a web interface for bioinformatics software) requiring scripting in bash and python. The github repo for that project can be found at the url \httplink{https://github.com/smilefreak/OtagoGalaxy}, this project remained a common fixture throughout all subsequent employment. Also helped members of the Merriman Lab with statistical and programming based questions writing scripts and adding tools to the galaxy instance as required.
\\
\\
Outputs
\begin{itemize}
\item Wrapped or created tools at the request of lab members and some faculty to the galaxy instance. The main languages used were bash and python with some C and Java. 
\item Server setup and administration of the galaxy server, requiring extensive unix command-line usage.  
\end{itemize}}

\cventry{2011-2013}{Laboratory teaching assistant}{Computer Science Department}{University Of Otago}{}{
Assisted with laboratories for first and second year over three years focusing on algorithms, datastructures and Object-orientated programming. C and Java were the languages used for these papers. 
}
\cventry{2011-2013}{Private Tutoring}{Computer Science}{University of Otago}{}{
Tutored two students through many computer science papers in the second and third year of their computer science degree. Java, C and SQL languages were required.
}
%------------------------------------------------

\section{Academic Outputs}
\cvitem{2013}{In preparation for publication: A\emph{M Cadzow, J Boocock, HT Nguyen, P Wilcox, TR Merriman and Michael A Black} bioinformatics workflow for detecting signatures of selection in genomic data (manuscript in preparation)} 
\cvitem{2013}{\emph{J Boocock, TR Merriman, MA Black and D Chagne}, Copy number variation in Malus x domestica (apple). Mapnet NZ Conference (Lincoln October 2013). }
\cvitem{2013}{\emph{J Boocock, D Eyers, P Wilcox, TR Merriman and MA Black}, Connecting Genetics Researchers to NeSI, eResearchNZ Conference (Christchurch July 2013).}


%----------------------------------------------------------------------------------------
%	COMPUTER SKILLS SECTION
%----------------------------------------------------------------------------------------


%----------------------------------------------------------------------------------------
%	COMMUNICATION SKILLS SECTION
%----------------------------------------------------------------------------------------



%----------------------------------------------------------------------------------------
%	LANGUAGES SECTION
%----------------------------------------------------------------------------------------

%----------------------------------------------------------------------------------------
%	INTERESTS SECTION
%----------------------------------------------------------------------------------------

\section{Interests}

\renewcommand{\listitemsymbol}{-~} % Changes the symbol used for lists

\cvlistdoubleitem{Guitar}{Weight Lifting}
\cvlistdoubleitem{Open source Software}{Reading}
\cvlistdoubleitem{Crowd Sourcing}{Data re-use}
%----------------------------------------------------------------------------------------
%	COVER LETTER
%----------------------------------------------------------------------------------------

% To remove the cover letter, comment out this entire block
\iffalse
\clearpage

\recipient{HR Departmnet}{Corporation\\123 Pleasant Lane\\12345 City, State} % Letter recipient
\date{\today} % Letter date
\opening{Dear Sir or Madam,} % Opening greeting
\closing{Sincerely yours,} % Closing phrase
\enclosure[Attached]{curriculum vit\ae{}} % List of enclosed documents

\makelettertitle % Print letter title

\lipsum[1-3] % Dummy text

\makeletterclosing % Print letter signature
\fi
%----------------------------------------------------------------------------------------

\end{document}
