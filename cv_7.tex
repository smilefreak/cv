%%%%%%%%%%%%%%%%%%%%%%%%%%%%%%%%%%%%%%%%%
% "ModernCV" CV and Cover Letter
% LaTeX Template
% Version 1.1 (9/12/12)
%
% This template has been downloaded from:
% http://www.LaTeXTemplates.com
%
% Original author:
% Xavier Danaux (xdanaux@gmail.com)
%
% License:
% CC BY-NC-SA 3.0 (http://creativecommons.org/licenses/by-nc-sa/3.0/)
%
% Important note:
% This template requires the moderncv.cls and .sty files to be in the same 
% directory as this .tex file. These files provide the resume style and themes 
% used for structuring the document.
%
%%%%%%%%%%%%%%%%%%%%%%%%%%%%%%%%%%%%%%%%%

%----------------------------------------------------------------------------------------
%	PACKAGES AND OTHER DOCUMENT CONFIGURATIONS
%----------------------------------------------------------------------------------------

\documentclass[11pt,a4paper,sans]{moderncv} % Font sizes: 10, 11, or 12; paper sizes: a4paper, letterpaper, a5paper, legalpaper, executivepaper or landscape; font families: sans or roman

\moderncvstyle{casual} % CV theme - options include: 'casual' (default), 'classic', 'oldstyle' and 'banking'
\moderncvcolor{blue} % CV color - options include: 'blue' (default), 'orange', 'green', 'red', 'purple', 'grey' and 'black'

\usepackage{lipsum} % Used for inserting dummy 'Lorem ipsum' text into the template

\usepackage[scale=0.75]{geometry} % Reduce document margins
%\setlength{\hintscolumnwidth}{3cm} % Uncomment to change the width of the dates column
%\setlength{\makecvtitlenamewidth}{10cm} % For the 'classic' style, uncomment to adjust the width of the space allocated to your name

%----------------------------------------------------------------------------------------
%	NAME AND CONTACT INFORMATION SECTION
%----------------------------------------------------------------------------------------

\firstname{James} % Your first name
\familyname{Boocock} % Your last name

% All information in this block is optional, comment out any lines you don't need
\title{Curriculum Vitae}
\address{Apartment 3, 161 High Street}{Dunedin, New Zealand}
\mobile{+64221201361}
\email{smilefreak@gmx.com}

%----------------------------------------------------------------------------------------

\begin{document}

\makecvtitle % Print the CV title

%----------------------------------------------------------------------------------------
%	EDUCATION SECTION
%----------------------------------------------------------------------------------------

\section{Education}

\cventry{2012--2013}{Diploma for Graduates}{The University of Otago}{}{}{Genetics and Statistics}  % Arguments not required can be left empty
\cventry{2008--2012}{Bachelor of Science}{The University of Otago}{}{}{Computer Science}

%----------------------------------------------------------------------------------------
%	WORK EXPERIENCE SECTION
%----------------------------------------------------------------------------------------

\section{Work Experience}

\cventry{Mid-2012--Present}{Assosciate Research Fellow}{Biochemistry Department}{University of Otago}{}{The research focused on assessing Copy-number variation in plant and human DNA sequence. 
\\
\\
Current Achievements
\begin{itemize}
\item Custom XML-RPC server in python for processing 1000genomes data on the amazon cloud for further processing on local HPC instances. 
\item Novel Permutation Approach for assessing significance of polymorphic regions using R.
\item Large data-set processing using unix command-line tools.
\end{itemize}}

%------------------------------------------------

\cventry{2012--2013}{Summer Studentship}{Summer Of eResearch}{Biochemistry Department}{University of Otago}{Worked on an interface between galaxy and Globus using python. Other work from the project involved developing a bioinformatics pipeline for analyses of selection signatures. The pipeline is currently in preparation for publication and can be found. \href{https://github.com/smilefreak/MerrimanSelectionPipeline}{here}
\\
\\
Achievements
\begin{itemize}
\item Development version of a galaxy globus interface using grython a jython version of Grisu which is a open source java framework designed to sit on top of grid middleware. 
\item Parralelisation of some Selection Tools for standard use and scripts to take advantage of the NESI IBM load leveler instance.
\item Python program to run selection\_tools and processing the intermediate datafiles, in preparation for publication.
\end{itemize}}
\cventry{2012--2013}{Summer StudentShip}{Summer Of eResearch}{Biochemistry Department}{University of Otago}{
The project focused on maintaining a local galaxy bioinformatics instance. Added a large amount of tools to the local galaxy instance requiring scripting in bash and python. The github repo for that project can be found \href{https://github.com/smilefreak/OtagoGalaxy}{here}. The project that was started over this summer has remained a common fixture throughout all subsequent employment, helping members of the biochemistry department specifically the Merriman Lab with statistical and programming based questions writing scripts and adding tools to the galaxy instance as required.
\\
\\
Achievements
\begin{itemize}
\item Either wrapped or created tools at the request of lab members and some faculty to the galaxy instance. Command languages were bash and python with some C and Java. 
\item Server setup and administration of the galaxy server, requiring extensive unix command-line usage.  
\end{itemize}}

\cventry{2011-2013}{Demonstrating}{Computer Science Department}{University Of Otago}{}{
Demonstrated three computer science papers from first and second year over three years focusing on algorithms and datastructures. C and Java were the languages used for these papers. 
}
\cventry{2011-2013}{Private Tutoring}{Computer Science}{University of Otago}{}{
Maintained a posistion tutoring two students through many computer science papers in the second and third year of their computer science degree.
}
%------------------------------------------------

\section{Academic}
\cvitem{2013}{In preparation Selection Pipeline Paper}
\cvitem{2013}{Oral Presentation at eRsesearch NZ 2013, Christchurch}
\cvitem{2013}{Oral Presentation at MapNet Meeting 2013, Lincoln}


%----------------------------------------------------------------------------------------
%	COMPUTER SKILLS SECTION
%----------------------------------------------------------------------------------------

\section{Skills}

\cvitem{Basic}{\textsc{SQL}}
\cvitem{Intermediate}{\textsc{Java},\textsc{C},\textsc{C++},\textsc{R},Linux System Administration, Parralel Programming,\textsc{IBM Load Leveler},Statistics}
\cvitem{Advanced}{Python, Unix Command-line}

%----------------------------------------------------------------------------------------
%	COMMUNICATION SKILLS SECTION
%----------------------------------------------------------------------------------------



%----------------------------------------------------------------------------------------
%	LANGUAGES SECTION
%----------------------------------------------------------------------------------------

%----------------------------------------------------------------------------------------
%	INTERESTS SECTION
%----------------------------------------------------------------------------------------

\section{Interests}

\renewcommand{\listitemsymbol}{-~} % Changes the symbol used for lists

\cvlistdoubleitem{Guitar}{Weight Lifting}
\cvlistdoubleitem{Opensource Software}{Reading}

%----------------------------------------------------------------------------------------
%	COVER LETTER
%----------------------------------------------------------------------------------------

% To remove the cover letter, comment out this entire block
\iffalse
\clearpage

\recipient{HR Departmnet}{Corporation\\123 Pleasant Lane\\12345 City, State} % Letter recipient
\date{\today} % Letter date
\opening{Dear Sir or Madam,} % Opening greeting
\closing{Sincerely yours,} % Closing phrase
\enclosure[Attached]{curriculum vit\ae{}} % List of enclosed documents

\makelettertitle % Print letter title

\lipsum[1-3] % Dummy text

\makeletterclosing % Print letter signature
\fi
%----------------------------------------------------------------------------------------

\end{document}
